\documentclass[letterpaper, 11pt]{article}
\usepackage{enumitem, titlesec}
\usepackage[colorlinks=true, urlcolor=black]{hyperref}
\usepackage[margin=0.5in]{geometry}

% disable page number
\pagestyle{empty}
% section setting
\titleformat{\section}{\scshape\Large}{}{0em}{}[\titlerule]
% vsapce between subsection
\newcommand{\subsectionvspace}{\vspace{8pt}}
% line space of item same to default line space of the body paragraph
\setlist[itemize]{noitemsep, topsep=0pt}
% no automatic indentation at the beginning of the body paragraph
\setlength{\parindent}{0pt}


\begin{document}


\begin{minipage}[t]{0.3\textwidth}
    \vspace*{0\baselineskip}\textbf{\huge Tianrui Qi}
\end{minipage}%
\begin{minipage}[t]{0.7\textwidth}
    \raggedleft
    401 17th St NW, Atlanta, GA 30363
    \\
    tianrui.qi@icloud.com 
    $\cdot$ 
    % tianrui.qi@gatech.edu 
    % \\
    +1(518)961-3370
    % $\cdot$ 
    % +86 136 9650 0794
    \\
    LinkedIn: \href{https://www.linkedin.com/in/tianrui-qi/}{tianrui-qi}
    $\cdot$
    GitHub: \href{https://github.com/tianrui-qi}{tianrui-qi}
\end{minipage}


\section{Education}


    \textbf{B.S. in Computer Science \hfill 01/2023 - 05/2025} \\
    \textit{Georgia Institute of Technology, Atlanta, GA}
    \begin{itemize}
        \item GPA: 3.92/4.00
        \item Achievements: President's Undergraduate Research Awards
    \end{itemize}

    \subsectionvspace

    \textbf{B.S. in Computer Science; Double Major in Mathematics \hfill 09/2020 - 12/2022} \\
    \textit{Rensselaer Polytechnic Institute, Troy, NY}
    \begin{itemize}
        \item GPA: 3.73/4.00
        \item Achievements: Dean's Honor List (every semester)
        \item Minor: Economics
    \end{itemize}


\section{Experience}


    \textbf{Undergraduate Research Assistant \hfill 04/2023 - Present} \\
    \textit{Georgia Institute of Technology and Emory University, Atlanta, GA} \\
    \textit{Jia Laboratory for Systems Biophotonics, PI: Shu Jia, Ph.D.}
    \begin{itemize}
        \item Engineered a scalable 3D U-Net pipeline based entirely on simulated data for volumetric localization in single-molecule super-resolution microscopy, resolving sub-cellular structure down to 60 nm.
        \item Developed a patch-based prediction pipeline that flexibly adapts to various input volume sizes and achieves a 100x speedup over conventional deterministic localization methods.
        \item Integrated the redundant cross-correlation algorithm for drift correction with the deep learning-based prediction pipeline.
    \end{itemize}

    \subsectionvspace

    \textbf{Co-op \hfill 01/2024 - 08/2024} \\
    \textit{Regeneron Genetics Center, Tarrytown, NY} \\
    \textit{Therapeutic Area Genetics, Manager: Jing He, Ph.D.}
    \begin{itemize}
        \item Utilized BERT-based large language models (LLMs) and unsupervised feature selection to obtrain a vector representation in a bio-meaningful space for each whole exome sequencing (WXS) sample.
        \item Demonstrated that the representations capture sample-wise differences by predicting immune system indicators of The Cancer Genome Atlas Program (TCGA) skin cancer samples.
        \item Scaled up the pipeline to handle hundreds WXS samples with billion DNA sequences by optimizing parallel computing for high-performance computing (HPC) and enhancing file system efficiency through hashing.
    \end{itemize}

    \subsectionvspace

    \textbf{Undergraduate Research Assistant \hfill 11/2021 - 12/2022} \\
    \textit{Rensselaer Polytechnic Institute, Troy, NY} \\
    \textit{AI-based X-ray Imaging System Lab, PI: Ge Wang, Ph.D.}
    \begin{itemize}
        \item Derived backward propagation formulation for quadratic neural networks and compared forward and backward propagation between quadratic and conventional neural networks mathematically.
        \item Implemented forward propagation, backward propagation, and training process of quadratic and conventional neural networks explicitly using NumPy in Python.
        \item Demonstrated that single-layer quadratic neural networks rival conventional neural networks with hundreds of neurons in classifying simulated and real-world Gaussian mixture data.
    \end{itemize}

    \subsectionvspace

    \textbf{Undergraduate Teaching Assistant \hfill 09/2022 - 12/2022} \\
    \textit{Rensselaer Polytechnic Institute, Troy, NY} \\
    \textit{Foundations of Computer Science, Instructor: David Goldschmidt, Ph.D.}
    \begin{itemize}
        \item Led weekly recitation sessions to help students understand course material.
        \item Assisted students' understanding of weekly lab exercises and graded assignments and exams.
    \end{itemize}


\section{Publications}


    \begin{minipage}[t]{0.9\textwidth} 
        Keyi Han$^\dag$, Xuanwen Hua$^\dag$, \textbf{Tianrui Qi}$^\dag$, Zijun Gao, Xiaopeng Wang, Shu Jia. 
        ``Volumetric Reconstruction and Localization Networks for 3D Single-molecule Localization Microscopy.''
        \textit{Manuscript in Preparation}.
    \end{minipage}%
    \begin{minipage}[t]{0.1\textwidth} \raggedleft
        \textbf{2025}
    \end{minipage}

    \subsectionvspace

    \begin{minipage}[t]{0.9\textwidth} 
        \textbf{Tianrui Qi}, Ge Wang. 
        ``Superiority of quadratic over conventional neural networks for classification of gaussian mixture data.'' 
        \textit{Visual Computing for Industry, Biomedicine, and Art}.
    \end{minipage}%
    \begin{minipage}[t]{0.1\textwidth} \raggedleft
        \textbf{2022}
    \end{minipage}

    \subsectionvspace

    \textit{$\dag$ denotes co-first authors}


\section{Course Projects}


    \textbf{Datapath and Control for a Turing Complete ISA with Interrupt Handling \hfill 09/2023 - 12/2023} \\
    \textit{Georgia Institute of Technology, Atlanta, GA} \\
    \textit{Introduction to Systems and Networking, Instructor: Daniel Forsyth}
    \begin{itemize}
        \item Designed a single-bus datapath and an efficient four-ROM microcontroller for a Turing complete instruction set architecture (ISA), supporting arithmetic, logical, memory access, and control flow instructions.
        \item Handled basic and input device interrupts by additional hardware including new instructions, interrupt registers, signals, and I/O bus, along with software supports such as interrupt vector tables.
    \end{itemize}

    \subsectionvspace

    \textbf{Alternating Direction Method of Multipliers for Support Vector Machine \hfill 01/2022 - 05/2022} \\
    \textit{Rensselaer Polytechnic Institute, Troy, NY} \\
    \textit{Computational Optimization, Instructor: Yangyang Xu, Ph.D.}
    \begin{itemize}
        \item Formulated the primal and augmented dual optimization problems for support vector machine (SVM) objective and developed alternating direction method of multipliers (ADMM) solver.
        \item Implemented the ADMM solver in MATLAB and reported the primal and dual feasibility violations at each outer iteration for the testing datasets.
    \end{itemize}

    % \subsectionvspace

    % \textbf{Windows of Susceptibility Analysis for Brain Diseases \hfill 01/2022 - 05/2022} \\
    % \textit{Rensselaer Polytechnic Institute, Troy, NY} \\
    % \textit{Data Mathematics, Instructor: Kristin Bennett, Ph.D.}
    % \begin{itemize}
    %     \item Performed the windows of susceptibility analysis based on mouse data from a similar brain-in-a-dish model for mice using R with k-means clustering and principal component analysis (PCA).
    %     \item Analyzed the same sets of microcephaly-associated genes and Zika-associated genes and detected similar windows of susceptibility for Microcephaly and Zika-induced microcephaly in mice as in humans.
    % \end{itemize}

    % \subsectionvspace

    % \textbf{Full Gate-Level Circuit in C for a Reduced MIPS ISA \hfill 09/2021 - 12/2021} \\
    % \textit{Rensselaer Polytechnic Institute, Troy, NY} \\
    % \textit{Computer Organization, Instructor: Konstantin Kuzmin, Ph.D.}
    % \begin{itemize}
    %     \item Designed a datapath for a reduced MIPS instruction set architectures (ISA) that support I-type instructions including \verb|lw|, \verb|sw|, \verb|beq|, \verb|addi|, R-type including \verb|and|, \verb|or|, \verb|add|, \verb|sub|, \verb|slt|, \verb|jr|, and J-type including \verb|j|, \verb|jal|.
    %     \item Implemented the datapath through a full gate-level circuit in C, including components of the processor like memory, control, arithmetic logic unit (ALU), decoder, adder, multiplexor, etc.
    % \end{itemize}


\section{Skills}


    \textbf{Programming:} Python (PyTorch, NumPy, pandas), MATLAB, Java, C, C++, R, Swift (ARKit), Bash, MIPS. \\
    \textbf{Development Tools:} Git, Conda, VS Code, RStudio, JetBrains Suite, Android Studio, Xcode. \\
    \textbf{Computing Platforms:} Linux (Ubuntu), AWS (EC2, S3), HPC (Slurm). \\
    \textbf{Software:} LaTeX, ImageJ, Adobe Illustrator. \\
    \textbf{Laboratory:} optics and laser alignment, fluorescence imaging, fluorescence labeling, cell culture maintenance. \\
    \textbf{Communication:} English (Professional), Mandarin (Native).


\section{References}


    \begin{minipage}[t]{0.48\textwidth} 
        \textbf{Xuanwen Hua, Ph.D.} \\
        Postdoctoral Fellow \\
        Coulter Department of Biomedical Engineering \\
        Georgia Institute of Technology and Emory University \\
        313 Ferst Drive, Atlanta, GA 30332 \\
        x.hua@gatech.edu $\cdot$ +1(631)590-9121
    \end{minipage}%
    \begin{minipage}[t]{0.04\textwidth}
        \
    \end{minipage}%
    \begin{minipage}[t]{0.48\textwidth}
        \textbf{Shu Jia, Ph.D.} \\
        Associate Professor \\
        Coulter Department of Biomedical Engineering \\
        Georgia Institute of Technology and Emory University \\
        313 Ferst Drive, Atlanta, GA 30332 \\
        shu.jia@gatech.edu $\cdot$ +1(404)894-0290
    \end{minipage}

    \subsectionvspace

    \textbf{Ge Wang, Ph.D.} \\
    Clark-Crossan Chair Professor and Director of the Biomedical Imaging Center \\
    Biomedical Engineering Departments \\
    Rensselaer Polytechnic Institute \\
    110 8th Street, Troy, NY 12180 \\
    wangg6@rpi.edu $\cdot$ +1(518)698-2500


\end{document}
