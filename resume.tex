\documentclass[letterpaper, 10pt]{article}
\usepackage{enumitem, titlesec, hyperref}
\usepackage[margin=0.5in]{geometry}
\pagestyle{empty}

% section setting
\titleformat{\section}{\scshape\large}{}{0em}{}[\titlerule]
% vsapce between subsection
\newcommand{\subsectionvspace}{\vspace{6pt}}
% line space of item same to default line space of the body paragraph
\setlist[itemize]{noitemsep, topsep=0pt}
% no automatic indentation at the beginning of the body paragraph
\setlength{\parindent}{0pt}


\begin{document}


\begin{center}
    \textbf{\LARGE Tianrui (Eric) Qi} \\
    +1(518)-961-3370 
    $\cdot$ 
    tianrui.qi@gatech.edu \\ 
    Linkedin: \href{https://www.linkedin.com/in/tianrui-qi/}{tianrui-qi} 
    $\cdot$ 
    GitHub: \href{https://github.com/tianrui-qi}{tianrui-qi}
\end{center}


\section{Education}


    \textbf{B.S. in Computer Science \hfill 01.2023 - (05.2025)} \\
    \textit{Georgia Institute of Technology, Atlanta, GA} \hfill GPA: 3.92/4.00
    \begin{itemize}
        \item Concentration: Modeling and Simulation, Theory
        \item Minor: Physics
    \end{itemize}

    \subsectionvspace

    \textbf{B.S. in Computer Science, Mathematics \hfill 09.2020 - 12.2022} \\
    \textit{Rensselaer Polytechnic Institute, Troy, NY} \hfill GPA: 3.73/4.00 
    \begin{itemize}
        \item Concentration: Theory, Algorithms, and Mathematics; Applied Mathematics, Mathematics of Computation 
        \item Minor: Economics         
        \item Honors: Dean's Honor List (all five semesters)
    \end{itemize}


\section{Experience}


    \textbf{Startup Founder},
    CREATE-X Idea to Prototype \hfill \textbf{08.2024 - present} \\
    \textit{
        Mentor: Dr. Xuanwen Hua, 
        Postdoctoral Fellow, \\
        Wallace H. Coulter Department of Biomedical Engineering,
        Georgia Institute of Technology and Emory University
    }
    \begin{itemize}
        \item[$\circ$] Conceptualizing an AR platforms that simulate interactions with 2D surfaces in a 3D space, addressing the limitation of screens that only support 2D writing and drawing and VR apps that focus solely on fully 3D interactions.
        \item[$\circ$] Exploring the computational power of Apple's AR platforms utilizing ARKit, assessing the extent of realistic user-environment interactions possible with current resources, and gathering user feedback to identify potential applications. 
        \item[$\circ$] Developing an prototype in iPhone that transforms a 3D indoor space into 2D canvas for creation and then projects back to the space for viewing, with plans to expand to more complex environments and additional devices.
    \end{itemize}

    \subsectionvspace

    \textbf{Co-op}, 
    Regeneron Genetics Center \hfill \textbf{01.2024 - 08.2024} \\
    \textit{
        Manager: Dr. Jing He, 
        Mgr Integrative Translational Genetics, \\
        Therapeutic Area Genetics,
        Regeneron Genetics Center
    }
    \begin{itemize}
        \item Utilized a BERT-based LLM to map DNA sequences in whole exome sequencing (WXS) samples into a bio-meaningful vector space and performed unsupervised feature selection to obtain a vector representation for each WXS sample.
        \item Demonstrated the representations capture sample-wise differences in somatic immune phenotypes by training the pipeline on 23 WXS samples and achieving 76\% accuracy in predicting leukocyte fraction on 17 TCGA SKCM samples.
        \item Scaled up the pipeline to handle about 1,000 WXS samples with 100 billion DNA sequences by optimizing parallel computing for HPC (Slurm) and enhancing file system efficiency through hashing.
    \end{itemize}

    \subsectionvspace

    \textbf{Undergraduate Research Assistant}, 
    Jia Laboratory for Systems Biophotonics \hfill \textbf{04.2023 - present} \\
    \textit{
        Principal Investigator: Dr. Shu Jia, 
        Associate Professor, \\
        Wallace H. Coulter Department of Biomedical Engineering,
        Georgia Institute of Technology and Emory University
    }
    \begin{itemize}
        \item Engineered a scalable 3D U-Net and training pipeline based entirely on simulated data for multi-scale super-resolution volumetric localization in single-molecule localization microscopy, achieving precise localization down to 20nm.
        \item Developed a patch-based prediction pipeline that flexibly adapts to various input volumes, requires minimal computational resources while easy to scale up, and achieves a 100x speedup over traditional Gaussian localization.
        \item Integrated the redundant cross-correlation algorithm for drift calculation and correction with the deep learning-based prediction pipeline, while balancing resource consumption and accuracy.
    \end{itemize}

    \subsectionvspace

    \textbf{Undergraduate Teaching Assistant}, 
    CSCI 2200 Foundations of Computer Science \hfill \textbf{09.2022 - 12.2022} \\
    \textit{
        Instructor: Dr. David Goldschmidt, 
        Executive Officer, \\
        Department of Computer Science, 
        Rensselaer Polytechnic Institute
    }
    %\begin{itemize}
        %\item Led recitation session weekly, addressing students' questions and deepening their understanding of learning objects, including logic proofs, combinatorics, probability, computation models, finite automata, and Turing machines.
        %\item Graded assignments and exams, ensured understanding of learning concepts, and proctored exams. Attended weekly labs, inspected lab exercises, and offered real-time guidance on course content.
    %\end{itemize}

    \subsectionvspace

    \textbf{Undergraduate Research Assistant}, 
    AI-based X-ray Imaging System Lab \hfill \textbf{11.2021 - 12.2022} \\
    \textit{
        Principal Investigator: Dr. Ge Wang, 
        Clark \& Crossan Endowed Chair Professor and Director of Biomedical Imaging Center, \\
        Department of Biomedical Engineering, 
        Rensselaer Polytechnic Institute
    }
    \begin{itemize}
        \item Derived backward propagation formulation for quadratic neural networks and compared forward and backward propagation between quadratic and conventional neural networks mathematically. 
        \item Implemented forward propagation, backward propagation, and training process of quadratic and conventional neural networks at mathematical computation level in Python using NumPy.
        \item Demonstrated that single-layer quadratic neural networks rivals conventional one with hundreds of neurons in classifying simulated and real-world Gaussian mixture data, highlighting efficacy and efficiency of quadratic neurons.
        %\item Integrated active dendrites with quadratic neurons to develop quadratic dendritic networks capable of adapting to changing task contexts and continuous learning.
    \end{itemize}


\section{Publication}


    \textit{† denotes co-first authors}

    \subsectionvspace

    Keyi Han$^\dag$, Xuanwen Hua$^\dag$, \textbf{Tianrui Qi}$^\dag$, Zijun Gao, Xiaopeng Wang, Shu Jia, ``Volumetric Reconstruction and Localization Networks for 3D Single-molecule Localization Microscopy,'' \textit{manuscript in preparation} (expected 2024).

    \subsectionvspace

    \textbf{Tianrui Qi}, Ge Wang, ``Superiority of quadratic over conventional neural networks for classification of gaussian mixture data,'' \textit{Visual Computing for Industry, Biomedicine, and Art} (2022).


\section{Academic Projects}


    \textbf{\href{https://github.com/tianrui-qi/ADMM-for-SVM}{Alternating Direction Method of Multipliers for Support Vector Machine}}, \\
    MATP 4820 Computational Optimization \hfill \textbf{01.2022 - 05.2022} \\
    \textit{
        Instructor: Dr. Yangyang Xu, 
        Associate Professor, \\
        Department of Mathematical Sciences, 
        Rensselaer Polytechnic Institute
    }
    \begin{itemize}
        \item Formulated the primal and augmented dual optimization problems for support vector machine (SVM) objective and developed alternating direction method of multipliers (ADMM) solver by solving two sub-optimization problems.
        \item Implemented the ADMM solver in MATLAB and reported the primal and dual feasibility violation at each outer iteration for the testing datasets.
    \end{itemize}

    \subsectionvspace

    \textbf{\href{https://github.com/tianrui-qi/WOS-Analysis}{Windows of Susceptibility Analysis for Brain Diseases}}, 
    MATP 4400 Data Mathematics \hfill \textbf{01.2022 - 03.2022} \\
    \textit{
        Instructor: Dr. Kristin Bennett, 
        Associate Director of Institute of Data Exploration and Applications, \\
        Department of Mathematical Sciences, 
        Rensselaer Polytechnic Institute
    }
    \begin{itemize}
        \item Performed the windows of susceptibility analysis based on mouse data from a similar brain-in-a-dish model for mice using R with machine learning techniques, including k-means clustering and principal component analysis (PCA).
        \item Analyzed the same sets of microcephaly-associated genes and Zika-associated genes and detected similar windows of susceptibility for Microcephaly and Zika-induced microcephaly in mice as in humans.
    \end{itemize}

    \subsectionvspace

    \textbf{\href{https://github.com/tianrui-qi/MIPS-Processor}{MIPS Processor in C}},
    CSCI 2500 Computer Organization \hfill \textbf{09.2021 - 12.2021} \\
    \textit{
        Instructor: Dr. Konstantin Kuzmin,
        Lecturer, \\
        Department of Computer Science,
        Rensselaer Polytechnic Institute
    }
    \begin{itemize}
        \item Represented the datapath for a reduced MIPS instruction set architectures (ISA) through a full gate-level circuit in C and implemented components of the processor, including memory, control, ALU, decoder, adder, multiplexor, etc.
        \item Supported I-type instructions including \verb|lw|, \verb|sw|, \verb|beq|, \verb|addi|, R-type instructions including \verb|and|, \verb|or|, \verb|add|, \verb|sub|, \verb|slt|, \verb|jr|, and J-type instructions \verb|j|, \verb|jal|.
    \end{itemize}


\section{Skills}


    \textbf{Programming Languages:} Python (PyTorch, NumPy, pandas), MATLAB, Java, C, C++, R, Swift (ARKit), Bash, MIPS
    \\
    \textbf{Development Tools:} Git, Conda, VSCode, JetBrains (PyCharm, IntelliJ, CLion, Android Studio), RStudio, Xcode
    \\
    \textbf{Computing Plantforms:} Linux (Ubuntu), AWS (EC2, S3), HPC (Slurm)
    \\
    \textbf{Software:} LaTeX, ImageJ, Adobe (Illustrator)
    \\
    \textbf{Communication:} English (Proficient), Mandarin (Native)


\end{document}
